\documentclass[11pt,a4paper,english]{article} % document type and language

\usepackage{babel}   % multi-language support
\usepackage{float}   % floats
\usepackage{url}     % urls
\usepackage{graphicx}
\graphicspath{{./data/}}

\begin{document}
	
	\section{Literature Overview}
		
	\subsection{Bergholt notes}
	\subsubsection{The Household}
	Optimal consumption decision \\
	\[
	max_{c_{it}} = (
	\int_{0}^{1} C_{it}^{\frac{\epsilon-1}{\epsilon}} di
	)^{\frac{\epsilon}{\epsilon-1}}
	\]
	s.t.
	
	\subsection{Nelson Genvieve}
	Quantitative easing through the portfolio balance channel. Households, hold long- and short-term bonds and have a preferred ratio. Deviation from this preference ratio decrease their utility according to $\frac{\phi}{2}(\frac{B_t^L}{B_t^S} - \delta_B)^{\phi}$. Long-term bonds are perpetuities, thus exist infinitely unless removed by the government. Short-term bonds exist for one period only.\\
	
	Price of raw capital\\
	
	\subsection{Boehl et al.}
	Central bank setting interest rate
	\includegraphics[scale=.3]{boehl_etal_CB_interest.png} \\
	
	Quantitative easing included as AR(2) process of capital and bonds, both of which are purchased by the CB.
	
	
	\section{Quantitative Easing}
	
	\subsection{An introduction}
	Channels of quantitative easing\\
	Signalling: through an announcement of QE future interest rates are assumed to be lower by economic agents\\
	Portfolio balance: Demand for long-term maturity debt or riskier assets increases, as they are being exchanged for short-term low interest rate debt. This decreases the interest rate on those assets now in higher demand as well.\\
	Liquidity: Liquidity is more broadly available to the market, and premia on liquidity are lowered. \\
	
	\section{Approaches to my thesis}
	
	\subsection{Including QE}
	Liquidity channel: Simplest way of including quantitative easing shocks is through stochastic process (possibly with drift) on bond holdings.
	
	\subsection{Including energy cost}
	Simple way of including energy price shocks is a homogenous price increase across goods. 
	\begin{itemize}
		\item Is a homogenous price increase through an energy shock a viable assumption?
	\end{itemize}

	\section{Thesis Outline}
	\subsection{Formal requirements}
	
	\subsection{Table of contents}
	Introduction\\
	Quantitative Easing
	\begin{itemize}
		\item Literature
		\item Applications and historic overview
		\item Interaction with energy price shock
	\end{itemize}
	Literature Overview DSGE Model for QE
	\begin{itemize}
		\item Applications of DSGE and QE
		\item criticism and short comings of DSGE in QE context
	\end{itemize}
	This thesis DSGE Model
	\begin{itemize}
		\item Structure
		\item Results
	\end{itemize}
	Conclusion and Discussion
	
\end{document}